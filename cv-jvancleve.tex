\documentclass[11pt]{article}

% hyperref and bibentry conflict workaround
\makeatletter
\let\saved@bibitem\@bibitem
\makeatother

\usepackage{bibentry}
\usepackage{natbib}
\usepackage{hyperref,doi}
\usepackage[nodayofweek]{datetime}

\usepackage{array}
\usepackage{booktabs}
\aboverulesep=1.5ex
\belowrulesep=1.5ex


\usepackage{enumitem}
\setlist[description, 1]{ itemsep=0.0ex, parsep=1ex }
\setlist[itemize, 1]{ label=, leftmargin=0ex, itemsep=0.0ex, parsep=1ex }
\setlist[itemize, 2]{ label=--, leftmargin = *, itemsep=0.0ex, parsep=0.5ex }
\newlist{publist}{itemize}{1}
\setlist[publist]{ label=, leftmargin=2ex, itemindent=-2ex, itemsep=0.0ex, parsep=2ex }
\newlist{compactlist}{itemize}{1}
\setlist[compactlist]{ label=, leftmargin=2ex, itemindent=-2ex, itemsep=0.0ex, parsep=0.5ex }

%%%%%%%%%%%%%%%%%%%%%%%%%%%%%%%%%%%%%%%%%%%%%%%%%%%%%%%%%%%%%%%%% 
% XeLaTeX font settings: COMMENT OUT IF NOT RUNNING XeLaTeX
\usepackage{fontspec}
\defaultfontfeatures{Ligatures=TeX}
\setmainfont{Adobe Garamond Pro}[BoldFont={AGaramondPro-Semibold}]
\setsansfont[Scale=MatchLowercase]{Myriad Pro}
\setmonofont[Scale=MatchUppercase]{Source Code Pro}
%%%%%%%%%%%%%%%%%%%%%%%%%%%%%%%%%%%%%%%%%%%%%%%%%%%%%%%%%%%%%%%%% 

\usepackage[letterpaper]{geometry}
\geometry{hmargin={1in,1in},vmargin={1in,1in}}
\hypersetup{
  colorlinks,
  urlcolor=black,
  pdftitle={Curriculum Vitae},
  pdfauthor={Jeremy Van Cleve},
  pdfcreator={XeLaTeX},
  pdfsubject={Curriculum Vitae},
  pdfkeywords={evolutionary biology ecology population genetics theory}
}

\usepackage{amsmath}
\usepackage{fancyhdr}
\fancypagestyle{plain}{%
 \fancyhf{} % clear all header and footer fields
 \fancyfoot[R]{\footnotesize \it last modified \today} % put last modified in lower right
 \renewcommand{\headrulewidth}{0pt}
 \renewcommand{\footrulewidth}{0pt}}
\pagestyle{fancy}
\fancyhead[L,C]{}
\fancyhead[R]{\footnotesize \bfseries \scshape Jeremy Van Cleve}
\renewcommand{\headrulewidth}{0.0pt}
\fancyfoot[L,R]{}
\fancyfoot[C]{\thepage}

\usepackage[parfill]{parskip}

\usepackage{calc}
\usepackage{marginnote}
\reversemarginpar
\usepackage[explicit]{titlesec}
\titleformat{\section}{\addfontfeature{LetterSpace=10.0} \Large \bfseries \scshape}{}{0ex}{#1}[]
\titlespacing*{\section}{-2ex}{1.0ex plus 0.5ex minus 0.5ex}{0.25ex plus 0.25ex minus 0.25ex}
% \titleformat{\subsection}{\bfseries \itshape}{\thesubsection.}{1ex}{#1}
% \titlespacing{\subsection}{0pt}{1ex plus 0.25ex minus 0.25ex}{0.0ex plus 0.25ex minus 0.25ex}
% \titleformat{\subsubsection}[runin]{\bfseries}{\roman{subsubsection}}{1ex}{#1.}
% \titlespacing{\subsubsection}{0pt}{1.5ex plus 0.5ex minus 0.25ex}{1em}

%% ===========================================================================%%

\begin{document}

% hyperref and bibentry conflict workaround
% \begingroup
% \makeatletter
% \let\@bibitem\saved@bibitem
\nobibliography{jv_pubs}
% \endgroup
\bibliographystyle{cv_prsb}

% ------------------------------------------------------------------------------%

\thispagestyle{plain}

\begin{center}
  \begin{tabular*}{6.45in}{p{1.75in}@{\hspace{0.25in}}p{2.45in}@{\hspace{0.25in}}p{1.75in}}
    & \begin{minipage}{2.45in} \centering \addfontfeature{LetterSpace=10.0} \Large \scshape \bfseries Jeremy Van Cleve \end{minipage}
    & \\[1.5em]
    \toprule
    \begin{minipage}{1.75in} \raggedleft
      {\href{http://vancleve.theoretical.bio}{vancleve.theoretical.bio} \footnotesize \scshape [web]}\\
      {\href{mailto:jvancleve@uky.edu}{jvancleve@uky.edu} \footnotesize \scshape [email]} \\
      {\href{http://twitter.com/jeremyvancleve}{@jeremyvancleve} \footnotesize \scshape [twitter]} \\
    \end{minipage}
    & \begin{minipage}{2.45in} \centering
      Department of Biology \\
      University of Kentucky \\
      Lexington, KY 40506
    \end{minipage}
    & \begin{minipage}{1.75in} \raggedright
      {(859) 218-3020 \footnotesize \scshape [phone]} \\
      {(859) 257-1717 \footnotesize \scshape [fax]}
    \end{minipage}\\
    \bottomrule
  \end{tabular*}
\end{center}

  % ------------------------------------------------------------------------------%

  \section{Education}

  \begin{center}
  \begin{tabular}{@{}p{0.5in}@{}p{1.5in}@{}p{2.5in}r}
    \textbf{Ph.D.} & \textbf{Stanford University} & Department of Biological Sciences & 2009 \\
    & Stanford, CA & Advisor: Professor Marcus W. Feldman & \\[2ex]
    \textbf{B.A.} & \textbf{Oberlin College} & Majors in Mathematics and Biology & 2003 \\
    & Oberlin, OH & &  
  \end{tabular}
\end{center}

  \section{Professional Experience}

  \begin{tabular}{@{}>{\raggedleft\arraybackslash}p{2.5in}@{\hspace{0.25in}}p{2in}r}
    \bfseries University of Kentucky & \textbf{Associate Professor of Biology} & 2021 -- \\
    \hspace{0.5in} Lexington, KY & & \\[2ex]

    \bfseries University of Kentucky & \textbf{Assistant Professor of Biology} & 2015 -- 2021 \\
    \hspace{0.5in} Lexington, KY & & \\[2ex]

    \bfseries National Evolutionary Synthesis Center & \textbf{NESCent Postdoctoral Fellow} & 2013--2014 \\
    \bfseries Duke University & & \\
    \hspace{0.5in} Durham, NC & & \\[2ex]

    \bfseries Santa Fe Institute & \textbf{SFI Omidyar Fellow} & 2009--2012 \\
    \hspace{0.5in} Santa Fe, NM &  & \\[2ex]

    \bfseries University of Colorado at Boulder & \textbf{Professional Research Assistant} & 2003--2004 \\
    \hspace{0.5in} Boulder, CO & &
  \end{tabular}

  \section{Publications {\small (*=co-first author)}}
  
  \begin{publist}
    
    \raggedright

\item \bibentry{AndrisJUrbanAff2021}

\item \bibentry{Dunoyer2020JEZB}

\item \bibentry{Akcay2020NormBook}

\item \bibentry{Van-Cleve2020TPB50th}

\item \bibentry{Estrela2019TREE}

\item \bibentry{Van-Cleve2017ICB}

\item \bibentry{Lehmann2016Evol}

\item \bibentry{WilkinsBioEssays2016}

\item \bibentry{Van-CleveJTB2016} 

\item \bibentry{AkcayPhilTrans2016}

\item \bibentry{AkcayCurrentOp2015}

\item \bibentry{Van-Cleve2015PNAS}

\item \bibentry{Van-Cleve2015TPB}

\item \bibentry{Servedio2014PLoSBiol}

\item \bibentry{Van-Cleve2014Evol}

\item \bibentry{Van-Cleve2013TPB}

\item \bibentry{Akcay2012AmNat}

\item \bibentry{Brandvain2011TiG}

\item \bibentry{Liberman2011Genetics}

\item \bibentry{Van-Cleve2010AmNat}

\item \bibentry{Akcay2009PNAS}

\item \bibentry{Salathe2009Genetics}

\item \bibentry{Van-Cleve2008JMB}

\item \bibentry{Van-Cleve2007Genetics}

\item \bibentry{Guralnick2005DivDist}

    
  \end{publist}
  
  \section{Preprints}
  
  \begin{publist}

    \item \bibentry{Thomson2022BgsCoop}

    \item \bibentry{HokeCriticalPeriod}
    
    \item \bibentry{Kessinger2018Draft}
        
    \item \bibentry{AndrisSocialCapital2016}
    
  \end{publist}
  
  \section{Grants}
  \begin{description}
  \item[2020--2024] Co-PI. National Science Foundation (DEB \#1953223). \$674,744\\
    ``Uneasy alliances: emergent properties and feedback mechanisms among manipulative endosymbiotic communities''
  \item[2019--2024] PI. National Science Foundation (DEB \#1846260). \$781,397\\
    ``CAREER: Genetic architecture and the construction of complex social traits''
  \item[2015--2017] PI. National Academy of Sciences and Keck Foundation -- Futures Initiative. \$50,000\\
    ``Social evolutionary systems biology: Studying collective behavior by integrating social evolution theory with sociogenomics''
  \item[2015] Co-PI. National Academy of Sciences and Keck Foundation -- Futures Initiative. \$25,000\\
    ``Origin of multicellular development via the capture of a stochastic process''
  \item[2013] Co-PI. Frost Foundation. \$20,000 \\
    ``Quantifying the Impact of Mentorships on Human and Social Capital in Santa Fe New Mexico''
  \end{description}
  
  \section{Awards and Honors}

  \begin{description}
  \item[2019] NSF CAREER Award
  \item[2014] Institute for Genomic Biology (UIUC), Carl Woese Fellowship (declined)
  \item[2010] Samuel Karlin Prize in Mathematical Biology (Stanford University) 
  \item[2009] National Institutes of Health Postdoctoral Fellowship (declined)
  \item[2008] National Institutes of Health NLM Training Grant Appointment
  \item[2007] Stanford University Centennial Teaching Award
  \item[2006] Stanford University Department of Biological Sciences Excellence in Teaching Award
  \item[2004] Anne T. and Robert M. Bass Stanford University Graduate Fellowship
  \item[2002] Elected Phi Beta Kappa
  \item[2001] Barry M. Goldwater Scholarship
  \end{description}

  \section{Invited Talks}
  \begin{compactlist}
  \item Santa Fe Institute, July 2021.
  \item University of Tennessee, National Institute for Mathematical and Biological Synthesis, September 2019.
  \item Washington University, Department of Biology, November 2017.
  \item University of Kentucky, Department of Geography, November 2017.
  \item Georgia Institute of Technology, School of Biological Sciences, October 2016.
  \item University of Kentucky, Center for Ecology, Evolution and Behavior Annual Symposium (Keynote). May 2015.
  \item University of Pennsylvania, Department of Biology, April 2014.
  \item North Carolina State University, Biomathematics Graduate Program, April 2014
  \item Florida State University, Department of Biology.  January 2014.
  \item University of California, San Diego, Section of Ecology, Behavior and Evolution.  January 2014.
  \item University of Kentucky, Department of Biology. December 2013.
  \item Harvey Mudd College, Department of Biology. December 2012.
  \item % ``Prosocial preferences, conformity, and the evolution of behavior within and between groups'' \\
    University of Lausanne, Switzerland, Department of Ecology and Evolution. May 2012.
  \item % ``Prosocial preferences, conformity, and the evolution of behavior within and between groups'' \\
    Center for Nonlinear Studies, Los Alamos National Laboratory. April 2012.
  \item % ``Prosocial preferences, conformity, and the evolution of behavior within and between groups'' \\
    University of Colorado, Boulder, Department of Ecology and Evolutionary Biology. January 2012.
  \item % ``Prosocial preferences and the evolution of behavior within and between groups'' \\
    University of New Mexico, Computer Science Department. November 2010.
  \item % ``Evolution and epigenetics: genomic imprinting in mammals and stochastic switching in bacteria'' \\
    National Institute for Mathematical Biology and Synthesis, Knoxville, TN. March 2010.
  \end{compactlist}  

  \section{Conferences and Workshops}

  \begin{compactlist}
  \item Society for Mathematical Biology Epi-PDEE Mini-Conference. Virtual. February 2023.
  \item Workshop: ``Collective Resilience'', Complexity Science Hub Vienna. Organizer. October 2022.
  \item Evolution Conference, Cleveland, Ohio. Two posters. June 2022.
  \item Workshop: ``Constructing and Deconstructing Collectives'', Santa Fe Institute (virutal). Organizer. January 2022.
  \item Evolution Conference, Online. Concurrent Session Talk. August 2021.
  \item American Society of Naturalists Conference. Asilomar, California. Concurrent session talk. January 2020.
  \item Evolution Conference, Providence, Rhode Island. Concurrent session talk. June 2019.
  \item 2nd Joint Congress on Evolutionary Biology, Montpellier, France. Poster. August 2018.
  \item American Society of Naturalists Conference, Asilomar, California. Concurrent session talk. January 2018.
  \item Evolution Conference, Austin, Texas. Concurrent session talk. June 2017.
  \item Society for Integrative and Comparative Biology, New Orleans, Louisiana.\\
    Invited symposium talk. January 2017.
  \item Evolution Conference, Austin, Texas. Concurrent session talk. June 2016.
  \item National Academies Keck Futures Initiative, ``Collective Behavior'', Irvine, CA. November 2014.
  \item Evolution Conference, Raleigh, North Carolina. Concurrent session talk. June 2014.
  \item Toulouse Economics \& Biology Workshop, Institute for Advanced Study in Toulouse, France. Poster. May 2014.
  \item American Society of Naturalists, Asilomar, CA. Concurrent session talk. January 2014.
  \item Evolution Conference, Snowbird, Utah. Concurrent session talk. June 2013.
  \item Evolution Conference, Ottawa, Canada. Concurrent session talk. July 2012.
  \item Animal Behavior Society and Human Behavioral and Evolution Society Meetings, Albuquerque, NM.  \\
    Concurrent session talk.  June 2012.
  \item Ecological Society of America Meeting, Austin, TX. Concurrent session talk. August 2011.
  \item Evolution Conference, Portland, OR. Concurrent session talk. June 2010
  \item Evolution Conference, Minneapolis, MN. Concurrent session talk, June 2008.
  \item EVO-WIBO Conference, Port Townsend, WA. Session talk. April 2008. 
  \item Evolution Conference, Christchurch, New Zealand. Concurrent session talk. June 2007. 
  \end{compactlist}
  
  \section{Mentoring}

  \begin{itemize}
  \item University of Kentucky (UK)
    \begin{itemize}
    \item Postdoctoral Mentor: Dr. Taylor Kessinger (UK, 2016--2019); Dr. Daniel Priego Espinosa (UK, 2020--)
    \item Ph.D. Advisor: Rokeya Rahman (2022--); Elliott Greene (UK, 2020--); Kathryn Greene (UK, 2019--); Luc A. Dunoyer (UK, 2017--2019)
    \item Ph.D. Committee Member: Kim Cook (UK, 2023--); Daniel Plaugher (UK, 2021--2022); Mariah Donohue (UK, 2019--); Kara Jones (UK, 2017--); Kaylynne Glover (UK, 2019--2022); Megan Thomas (UK, 2018--2019); Emily E. Bendall (UK, 2016--2020)
    \item M.S. Committee Member: James Giordano (UK, 2018); M. Grayson McWhorter (UK, 2016--2020);
    \item Undergraduate Mentor: Chetas Chavda (UK, 2023--); Evan Yang (UK, 2019--2020); Haley Holtmann (UK, 2019); Lauren Lawless (UK, 2016--2017)
    \item High School Mentor: Jonah Hubert (PLDHS, 2023--); Maanasa Muthukrishnan (PLDHS, 2019); Parker Smith (PLDHS, 2018--2020); Max Bograd (PLDHS, 2017--2019); Blake Jaeger (PLDHS, 2016--2018); Hugh Ronald (Paul Laurence Dunbar High School (PLDHS), 2015--2016)
    \end{itemize}
  \item National Evolutionary Synthesis Center / Duke University
    \begin{itemize}
    \item Chloe Atwater, Research Assistant, Santa Fe Institute (Summer--Fall 2013).
    \end{itemize}
  \item Santa Fe Institute (SFI)
    \begin{itemize}
    \item SFI Research Experience for Undergraduates Mentor: Austen Mack-Crane (Brown University, 2012); Brecia Young (Harvard University, 2011); 
      Amalie McKee (Case Western Reserve University, 2010)
    \item High School Mentor. The Masters Program Charter High School, Santa Fe, NM (2010--2012)
    \end{itemize}
  \end{itemize}

  \section{Teaching}
  
  \begin{itemize}
  \item University of Kentucky (UK)
    \begin{itemize}
    \item Fall 2021, 2022. BIO 621. ``Data Wrangling and Visualization Using R''
    \item Fall 2017, 2019, Spring 2021. BIO 425 Undergraduate Seminar. ``Reconstructing Human History with Genetics''
    \item Fall 2016, 2018. BIO 770 Graduate Seminar.``Data Wrangling and Visualization Using R''
    \item Fall 2020, Spring 2016--2020, 2022. BIO 325. ``Ecology''
    \item Fall 2015, Spring 2023. BIO 770 Graduate Seminar. ``Dr. Pangloss reborn? Natural selection in evolution''
    \end{itemize}
  \item National Evolutionary Synthesis Center / Duke University
    \begin{itemize}
    \item Lecturer: Santa Fe Institute Summer Complexity and Modeling Program, \\
      Groton School, MA (Summer 2013)
    \end{itemize}
  \item Santa Fe Institute (SFI)
    \begin{itemize}
    \item Lecturer: Santa Fe Institute Complex Systems Summer School (2011--2012)
    \end{itemize}
  \end{itemize}
  
  % \renewcommand{\arraystretch}{1.1}
  % \begin{tabular}{@{$\ast$ }r@{}l@{\hspace{2ex}}l}
  %   2010 & & Samuel Karlin Prize in Mathematical Biology (Stanford University) \\ 
  %   2009 & & National Institutes of Health Postdoctoral Fellowship (declined) \\
  %   2008 &--2009 & National Institutes of Health NLM Training Grant Appointment \\
  % %   2008 & & American Society of Naturalists Graduate Student Travel Award for Evolution Meeting \\
  %   2007 & & Stanford University Centennial Teaching Award \\
  %   2006 & & Stanford University Department of Biological Sciences Excellence in Teaching Award \\
  %   2004 & & Anne T. and Robert M. Bass Stanford University Graduate Fellowship \\
  %   2002 & & Elected Phi Beta Kappa \\
  %   2001 & & Barry M. Goldwater Scholarship \\
  % %   2003 & & Oberlin College Jason Chicoine Memorial Prize in Biology \\
  % %   2002 & & Oberlin College Joshua Levitt Memorial Prize in Biology \\
  % %   2001 & & Oberlin Biology Department Sophomore Achievement Award \\
  % %   2001 & & Oberlin Chemistry Department Jewett Prize \\
  % %   1999&--2003 & Oberlin College John Stern Science Scholarship
  % \end{tabular}

  % ------------------------------------------------------------------------------%

  \section{Academic Service}
  \begin{itemize}
    \item Editor (2022--present), Theoretical Population Biology.
    \item Associate Editor (2019--2023), American Naturalist
    \item Associate Editor (2016--2021), Theoretical Population Biology.
    \item Reviewer: \textit{Nature}, \textit{Nature Ecology and Evolution}, \textit{Nature Communications}, \textit{Proceedings of the National Academy of Sciences}, \textit{Genetics}, \textit{Ecology Letters}, \textit{Theoretical Population Biology}, \textit{Proceedings of the Royal Society B}, \textit{Philosophical Transactions of the Royal Society B}, \textit{The American Naturalist}, \textit{Evolution}, \textit{Evolutionary Ecology}, \textit{Evolutionary Biology}, \textit{Biological Reviews}, \textit{Behavioral Ecology}, \textit{Heredity}, \textit{Journal of Heredity}, \textit{PLOS Computational Biology}, \textit{PLOS Genetics}, \textit{PLOS ONE}, \textit{Journal of Theoretical Biology}, \textit{Bulletin of Mathematical Biology}, \textit{Journal of the Royal Society Interface}, \textit{Royal Society Open Science}, \textit{BMC Evolutionary Biology}, \textit{Ecological Complexity}, \textit{Frontiers in Ecology and Evolution}, \textit{Frontiers in Psychology}, \textit{Evolutionary Human Sciences}
  \item Member of the American Society of Naturalists and the Society for the Study of Evolution.
  \end{itemize}
  
  % ------------------------------------------------------------------------------%

  \section{Professional Development}
  \begin{description}
    \item[2022] The Association for Graduate Enrollment Management: Webinar: Cultivating a Welcoming and Affirming Space for Students of Color in Graduate Programs
    \item[2021] NSF IGEN: Equity in Graduate Admissions Workshop
    \item[2021] University of Kentucky College of Medicine: Mentor Training Course
    \item[2021] University of Kentucky College of Arts and Sciences: Faculty Mentoring Workshop
  \end{description}

  % ------------------------------------------------------------------------------%

  \section{Media Coverage}
  \begin{itemize}
  \item \textit{Albuquerque Journal} Health section article, June 13, 2011.
  \end{itemize}

  % ------------------------------------------------------------------------------%

\end{document}

%% ===========================================================================%%
