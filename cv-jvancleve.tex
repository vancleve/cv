\documentclass[11pt]{article}

% hyperref and bibentry conflict workaround
\makeatletter
\let\saved@bibitem\@bibitem
\makeatother

\usepackage{bibentry}
\usepackage{natbib}
\usepackage{hyperref,doi}
\usepackage[nodayofweek]{datetime}

\usepackage{array}
\usepackage{booktabs}
\aboverulesep=1.5ex
\belowrulesep=1.5ex


\usepackage{enumitem}
\newlist{publist}{itemize}{1}
\setlist[description, 1]{ itemsep=0.0ex, parsep=1ex }
\setlist[itemize, 1]{ label=, leftmargin=0ex, itemsep=0.0ex, parsep=1ex }
\setlist[itemize, 2]{ label=--, itemsep=0.0ex, parsep=0.5ex }
\setlist[publist]{ label=, leftmargin=2ex, itemindent=-2ex, itemsep=0.0ex, parsep=2ex }

%%%%%%%%%%%%%%%%%%%%%%%%%%%%%%%%%%%%%%%%%%%%%%%%%%%%%%%%%%%%%%%%% 
% XeLaTeX font settings: COMMENT OUT IF NOT RUNNING XeLaTeX
\usepackage{fontspec}
\defaultfontfeatures{Ligatures=TeX}
\setmainfont{Adobe Garamond Pro}
\setsansfont[Scale=MatchLowercase]{Myriad Pro}
\setmonofont[Scale=MatchUppercase]{Source Code Pro}
%%%%%%%%%%%%%%%%%%%%%%%%%%%%%%%%%%%%%%%%%%%%%%%%%%%%%%%%%%%%%%%%% 

\usepackage[letterpaper]{geometry}
\geometry{hmargin={1in,1in},vmargin={1in,1in}}
\hypersetup{
  colorlinks,
  urlcolor=black,
  pdftitle={Curriculum Vitae},
  pdfauthor={Jeremy Van Cleve},
  pdfcreator={XeLaTeX},
  pdfsubject={Curriculum Vitae},
  pdfkeywords={evolutionary biology ecology population genetics theory}
}

\usepackage{amsmath}
\usepackage{fancyhdr}
\pagestyle{fancy}
\lhead{}
\chead{}
\rhead{\footnotesize \bfseries \scshape Jeremy Van Cleve}
\renewcommand{\headrulewidth}{0.0pt}
\lfoot{}
\cfoot{\thepage}
\rfoot{}

\usepackage[parfill]{parskip}

\usepackage{calc}
\usepackage{marginnote}
\reversemarginpar
\usepackage[explicit]{titlesec}
\titleformat{\section}{\addfontfeature{LetterSpace=10.0} \Large \bfseries \scshape}{}{0ex}{#1}[]
\titlespacing*{\section}{-2ex}{1.0ex plus 0.5ex minus 0.5ex}{0.25ex plus 0.25ex minus 0.25ex}
% \titleformat{\subsection}{\bfseries \itshape}{\thesubsection.}{1ex}{#1}
% \titlespacing{\subsection}{0pt}{1ex plus 0.25ex minus 0.25ex}{0.0ex plus 0.25ex minus 0.25ex}
% \titleformat{\subsubsection}[runin]{\bfseries}{\roman{subsubsection}}{1ex}{#1.}
% \titlespacing{\subsubsection}{0pt}{1.5ex plus 0.5ex minus 0.25ex}{1em}

%% ===========================================================================%%

\begin{document}

% hyperref and bibentry conflict workaround
% \begingroup
% \makeatletter
% \let\@bibitem\saved@bibitem
\nobibliography{jv_pubs}
% \endgroup
\bibliographystyle{cv_prsb}

% ------------------------------------------------------------------------------%

\thispagestyle{empty}

\begin{center}
  \begin{tabular*}{6.45in}{p{1.75in}@{\hspace{0.25in}}p{2.45in}@{\hspace{0.25in}}p{1.75in}}
    & \begin{minipage}{2.45in} \centering \addfontfeature{LetterSpace=10.0} \Large \scshape \bfseries Jeremy Van Cleve \end{minipage}
    & \\[1.5em]
    \toprule
    \begin{minipage}{1.75in} \raggedleft
      {\href{http://vancleve.theoretical.bio}{vancleve.theoretical.bio} \footnotesize \scshape [web]}\\
      {\href{mailto:jvancleve@uky.edu}{jvancleve@uky.edu} \footnotesize \scshape [email]} \\
      {\href{http://twitter.com/jeremyvancleve}{@jeremyvancleve} \footnotesize \scshape [twitter]} \\
    \end{minipage}
    & \begin{minipage}{2.45in} \centering
      Department of Biology \\
      University of Kentucky \\
      Lexington, KY 40506
    \end{minipage}
    & \begin{minipage}{1.75in} \raggedright
      {(859) 218-3020 \footnotesize \scshape [phone]} \\
      {(859) 257-1717 \footnotesize \scshape [fax]}
    \end{minipage}\\
    \bottomrule
  \end{tabular*}
\end{center}

  % ------------------------------------------------------------------------------%

  \section{Education}

  \begin{center}
  \begin{tabular}{@{}p{0.5in}@{}p{1.5in}@{}p{2.5in}r}
    \textbf{Ph.D.} & \textbf{Stanford University} & Department of Biological Sciences & 2009 \\
    & Stanford, CA & Advisor: Professor Marcus W. Feldman & \\[2ex]
    \textbf{B.A.} & \textbf{Oberlin College} & Majors in Mathematics and Biology & 2003 \\
    & Oberlin, OH & &  
  \end{tabular}
\end{center}

  \section{Professional Experience}

  \begin{tabular}{@{}>{\raggedleft\arraybackslash}p{2.5in}@{\hspace{0.25in}}p{2in}r}
    \bfseries University of Kentucky & \textbf{Assistant Professor of Biology} & 2015 -- \\
    \hspace{0.5in} Lexington, KY & & \\[2ex]

    \bfseries National Evolutionary Synthesis Center & \textbf{NESCent Postdoctoral Fellow} & 2013--2014 \\
    \bfseries Duke University & & \\
    \hspace{0.5in} Durham, NC & & \\[2ex]

    \bfseries Santa Fe Institute & \textbf{SFI Omidyar Fellow} & 2009--2012 \\
    \hspace{0.5in} Santa Fe, NM &  & \\[2ex]

    \bfseries University of Colorado at Boulder & \textbf{Professional Research Assistant} & 2003--2004 \\
    \hspace{0.5in} Boulder, CO & &
  \end{tabular}

  % \title{\bfseries \scshape Professional Research Assistant}
  % \dates{2003-2004}
  % \employer{University of Colorado at Boulder}
  % \location{Boulder, CO} 
  % \begin{position} \\[-4mm] 
  %   Studied bird species richness in Rocky Mountains using taxonomic databases with Professor Robert Guralnick \\
  %   and wrote simulations for single-molecule fluorescence experiments with Professor Arthur Pardi.
  % \end{position}

  % \title{Professional Research Assistant}
  % \dates{2003-2004}
  % \employer{University of Colorado at Boulder  \\ with Professor Arthur Pardi}
  % \location{Boulder, CO} 
  % \begin{position} \\[-4mm] 
  %   \\ Wrote software to simulate and analyze data from single-molecule fluorescence experiments.
  % \end{position} 

  % \title{NSF Research Experience for Undergrads}
  % \dates{Summer 2002}
  % \employer{University of Colorado at Boulder \\ with Professor Roger Pilke Jr.}
  % \location{Boulder, CO}
  % \begin{position} \\[-4mm] 
  %   \\ Studied the scientific, political, and philosophical elements of climate change.\\
  %   Created policy oriented analysis of the role of prediction in the Yucca Mountain Project. 
  % \end{position}

  % \title{NSF Research Experience for Undergrads}
  % \dates{Summer 2001}
  % \employer{The Santa Fe Institute with Dr. Ludo Pagie}
  % \location{Santa Fe, NM}
  % \begin{position} \\[-4mm] 
  %   Project title: “Diversity and evolutionary dynamics in an iron-deprived microbial community” \\
  %   Created spatially explicit and analytical simulations in order to study evolutionary dynamics. 
  % \end{position}

  % \title{Intern}
  % \dates{Summers 1999 \& 2000}
  % \employer{Los Alamos National Laboratory \\ with Dr. Catherine Macken}
  % \location{Los Alamos, NM}
  % \begin{position} \\[-4mm] 
  %   \\ Wrote phylogenetic software packages in Perl and revamped old packages for use over the Internet.
  % \end{position}

  % ------------------------------------------------------------------------------%

  \section{Publications {\small (*=co-first/corresponding author)}}
  
  \begin{publist}
    
    \raggedright

\item \bibentry{AndrisJUrbanAff2021}

\item \bibentry{Dunoyer2020JEZB}

\item \bibentry{Akcay2020NormBook}

\item \bibentry{Van-Cleve2020TPB50th}

\item \bibentry{Estrela2019TREE}

\item \bibentry{Van-Cleve2017ICB}

\item \bibentry{Lehmann2016Evol}

\item \bibentry{WilkinsBioEssays2016}

\item \bibentry{Van-CleveJTB2016} 

\item \bibentry{AkcayPhilTrans2016}

\item \bibentry{AkcayCurrentOp2015}

\item \bibentry{Van-Cleve2015PNAS}

\item \bibentry{Van-Cleve2015TPB}

\item \bibentry{Servedio2014PLoSBiol}

\item \bibentry{Van-Cleve2014Evol}

\item \bibentry{Van-Cleve2013TPB}

\item \bibentry{Akcay2012AmNat}

\item \bibentry{Brandvain2011TiG}

\item \bibentry{Liberman2011Genetics}

\item \bibentry{Van-Cleve2010AmNat}

\item \bibentry{Akcay2009PNAS}

\item \bibentry{Salathe2009Genetics}

\item \bibentry{Van-Cleve2008JMB}

\item \bibentry{Van-Cleve2007Genetics}

\item \bibentry{Guralnick2005DivDist}

    
  \end{publist}
  
  \section{Preprints}
  
  \begin{publist}

  \item \bibentry{AndrisBBBS}
    
  \item \bibentry{HokeCriticalPeriod}
    
  \item \bibentry{AndrisSocialCapital}
    
  \end{publist}
  

% \section{Manuscripts in preparation}

%   { \setlength{\parskip}{1.25ex}
%     
    
    % \bibentry{Van-ClevePlasticity}

    % \bibentry{Van-CleveImprintEpi}
% }

  \section{Grants}
  \begin{description}
  \item[2015] PI. National Academy of Sciences and Keck Foundation -- Futures Initiative. \$50,000\\
    ``Social evolutionary systems biology: Studying collective behavior by integrating social evolution theory with sociogenomics''
  \item[2015] Co-PI. National Academy of Sciences and Keck Foundation -- Futures Initiative. \$25,000\\
    ``Origin of multicellular development via the capture of a stochastic process''
  \item[2013] Co-PI. Frost Foundation. \$20,000 \\
    ``Quantifying the Impact of Mentorships on Human and Social Capital in Santa Fe New Mexico''
  \end{description}
  
  \section{Awards and Honors}

  \begin{description}
  \item[2014] Institute for Genomic Biology (UIUC), Carl Woese Fellowship (declined)
  \item[2010] Samuel Karlin Prize in Mathematical Biology (Stanford University) 
  \item[2009] National Institutes of Health Postdoctoral Fellowship (declined)
  \item[2008] National Institutes of Health NLM Training Grant Appointment
  \item[2007] Stanford University Centennial Teaching Award
  \item[2006] Stanford University Department of Biological Sciences Excellence in Teaching Award
  \item[2004] Anne T. and Robert M. Bass Stanford University Graduate Fellowship
  \item[2002] Elected Phi Beta Kappa
  \item[2001] Barry M. Goldwater Scholarship
  \end{description}


  \section{Invited Talks}
  \begin{itemize}
  \item Washington University, Department of Biology, November 2017.
  \item University of Kentucky, Department of Geography, November 2017.
  \item Georgia Institute of Technology, School of Biological Sciences, October 2016.
  \item University of Kentucky, Center for Ecology, Evolution and Behavior Annual Symposium (Keynote). May 2015.
  \item University of Pennsylvania, Department of Biology, April 2014.
  \item North Carolina State University, Biomathematics Graduate Program, April 2014
  \item Florida State University, Department of Biology.  January 2014.
  \item University of California, San Diego, Section of Ecology, Behavior and Evolution.  January 2014.
  \item University of Kentucky, Department of Biology. December 2013.
  \item Harvey Mudd College, Department of Biology. December 2012.
  \item % ``Prosocial preferences, conformity, and the evolution of behavior within and between groups'' \\
    University of Lausanne, Switzerland, Department of Ecology and Evolution. May 2012.
  \item % ``Prosocial preferences, conformity, and the evolution of behavior within and between groups'' \\
    Center for Nonlinear Studies, Los Alamos National Laboratory. April 2012.
  \item % ``Prosocial preferences, conformity, and the evolution of behavior within and between groups'' \\
    University of Colorado, Boulder, Department of Ecology and Evolutionary Biology. January 2012.
  \item % ``Prosocial preferences and the evolution of behavior within and between groups'' \\
    University of New Mexico, Computer Science Department. November 2010.
  \item % ``Evolution and epigenetics: genomic imprinting in mammals and stochastic switching in bacteria'' \\
    National Institute for Mathematical Biology and Synthesis, Knoxville, TN. March 2010.
  \end{itemize}
  
  \section{Conferences}

  \begin{itemize}
  \item American Society of Naturalists Conference, Concurrent session talk. January 2018.
  \item Evolution Conference, Austin, Texas. Concurrent session talk. June 2017.
  \item Society for Integrative and Comparative Biology, New Orleans, Louisiana. Invited symposium talk. January 2017.
  \item Evolution Conference, Austin, Texas. Concurrent session talk. June 2016.
  \item National Academies Keck Futures Initiative, ``Collective Behavior'', Irvine, CA. November 2014.
  \item Evolution Conference, Raleigh, North Carolina. Concurrent session talk. June 2014.
  \item Toulouse Economics \& Biology Workshop, Institute for Advanced Study in Toulouse, France. Poster. May 2014.
  \item American Society of Naturalists, Asilomar, CA. Concurrent session talk. January 2014.
  \item Evolution Conference, Snowbird, Utah. Concurrent session talk. June 2013.
  \item Evolution Conference, Ottawa, Canada. Concurrent session talk. July 2012.
  \item Animal Behavior Society and Human Behavioral and Evolution Society Meetings, Albuquerque, NM.  \\
    Concurrent session talk.  June 2012.
  \item % ``Behavioral responses in structured populations pave the way to group adaptation'' \\
    Ecological Society of America Meeting, Austin, TX. Concurrent session talk. August 2011.
  \item % ``Stochastic switching and the evolution of mutation in asymmetric fitness landscapes '' \\
    Evolution Conference, Portland, OR. Concurrent session talk. June 2010
  \item % ``The Evolution of Cooperation by Mutual Regard'' \\
    Evolution Conference, Minneapolis, MN. Concurrent session talk, June 2008. \\
    EVO-WIBO Conference, Port Townsend, WA. Session talk. April 2008. 
  \item % ``Sex-specific selection, sex-linkage and dominance in genomic imprinting'' \\
    Evolution Conference, Christchurch, New Zealand. Concurrent session talk. June 2007. 
  \end{itemize}

  \section{Teaching Experience}

  \begin{itemize}
  \item University of Kentucky (UK)
    \begin{itemize}
    \item Fall 2017. BIO 425 Undergraduate Seminar. ``Reconstructing Human History with Genetics''
    \item Fall 2016, 2018. BIO 770 Graduate Seminar.``Data Wrangling and Visualization Using R''
    \item Spring 2016, 2017, 2018. BIO 325. ``Ecology''
    \item Fall 2015. BIO 770 Graduate Seminar. ``Dr. Pangloss reborn? Natural selection in evolution''
    \item Postdoctoral Fellow Mentor: Dr. Taylor Kessinger (UK, 2016--)
    \item Ph.D. Advisor: Luc A. Dunoyer (UK, 2017--)
    \item Ph.D. Committee Member: Emily E. Bendall (UK, 2016--); M. Grayson McWhorter (UK, 2016--); Kara Jones (UK, 2017--)
    \item Undergraduate Mentor: Lauren Lawless (UK, 2016--)
    \item High School Mentor: Hugh Ronald (Paul Laurence Dunbar High School (PLDHS), 2015--2016); Blake Jaeger (PLDHS, 2016--); Max Bograd (PLDHS, 2017--)
    \end{itemize}
  \item National Evolutionary Synthesis Center / Duke University
    \begin{itemize}
    \item Lecturer: Santa Fe Institute Summer Complexity and Modeling Program, \\
      Groton School, MA (Summer 2013)
    \item Mentor: Chloe Atwater, Research Assistant, Santa Fe Institute (Summer--Fall 2013).
    \end{itemize}
  \item Santa Fe Institute (SFI)
    \begin{itemize}
    \item Lecturer: Santa Fe Institute Complex Systems Summer School (2011--2012)
    \item SFI Research Experience for Undergraduates Mentor: Austen Mack-Crane (Brown University, 2012); Brecia Young (Harvard University, 2011); 
      Amalie McKee (Case Western Reserve University, 2010)
    \item High School Mentor. The Masters Program Charter High School, Santa Fe, NM (2010--2012)
    \end{itemize}
  % \item Stanford University
  %   \begin{itemize}
  %   \item Biology Department Teaching Assistant Training Program: Electronic Resource Development; 2008--9
  %   \item Teaching Consultant for the Stanford Center for Teaching and Learning; 2007--9
  %   \item Guest Lecturer \& Teaching Assistant, Fundamentals of Molecular Evolution (Biosci 113/244); Spring 2008
  %   \item Guest Lecturer, Theoretical Population Genetics (Biosci 183); Winter 2008
  %   \item Teaching Assistant, Plant Biology, Evolution, and Ecology (Biosci 43); Spring 2006
  %   \item Teaching Assistant, Biostatistics (Biosci 141); Winter 2005
  %   \end{itemize}
  \end{itemize}
  
  % \renewcommand{\arraystretch}{1.1}
  % \begin{tabular}{@{$\ast$ }r@{}l@{\hspace{2ex}}l}
  %   2010 & & Samuel Karlin Prize in Mathematical Biology (Stanford University) \\ 
  %   2009 & & National Institutes of Health Postdoctoral Fellowship (declined) \\
  %   2008 &--2009 & National Institutes of Health NLM Training Grant Appointment \\
  % %   2008 & & American Society of Naturalists Graduate Student Travel Award for Evolution Meeting \\
  %   2007 & & Stanford University Centennial Teaching Award \\
  %   2006 & & Stanford University Department of Biological Sciences Excellence in Teaching Award \\
  %   2004 & & Anne T. and Robert M. Bass Stanford University Graduate Fellowship \\
  %   2002 & & Elected Phi Beta Kappa \\
  %   2001 & & Barry M. Goldwater Scholarship \\
  % %   2003 & & Oberlin College Jason Chicoine Memorial Prize in Biology \\
  % %   2002 & & Oberlin College Joshua Levitt Memorial Prize in Biology \\
  % %   2001 & & Oberlin Biology Department Sophomore Achievement Award \\
  % %   2001 & & Oberlin Chemistry Department Jewett Prize \\
  % %   1999&--2003 & Oberlin College John Stern Science Scholarship
  % \end{tabular}

  % ------------------------------------------------------------------------------%

  \section{Academic Service}
  \begin{itemize}
    \item Associate Editor (2016--2019), Theoretical Population Biology.
    \item Reviewer: \textit{Nature Communications}, \textit{Proceedings of the National Academy of Sciences}, \textit{Genetics}, \textit{Theoretical Population Biology}, \textit{Proceedings of the Royal Society B}, \textit{Philosophical Transactions of the Royal Society B}, \textit{The American Naturalist}, \textit{Evolution}, \textit{Evolutionary Ecology}, \textit{Evolutionary Biology}, \textit{Biological Reviews}, \textit{Behavioral Ecology}, \textit{Heredity}, \textit{PLoS Computational Biology}, \textit{PLoS ONE}, \textit{Journal of Theoretical Biology}, \textit{Bulletin of Mathematical Biology}, \textit{Journal of the Royal Society Interface}, \textit{Royal Society Open Science}, \textit{BMC Evolutionary Biology}, \textit{Ecological Complexity}.
  \item Member of the American Society of Naturalists and the Society for the Study of Evolution.
  \end{itemize}
  
  % ------------------------------------------------------------------------------%

  \section{Media Coverage}
  \begin{itemize}
  \item \textit{Albuquerque Journal} Health section article, June 13, 2011.
  \end{itemize}

  % ------------------------------------------------------------------------------%

  \pagestyle{fancy}
  \rfoot{\footnotesize \it last modified \today}

\end{document}

%% ===========================================================================%%
